\section{ Einführung }\label{se:intro}
\paragraph*{}
Ziel der Übung ist es, die drei Algorithmen A*, Breitensuche und IDA* auf ein Wegefindungsproblem in einem 
schlichten Digraphen $G = (V,A,c)$ anzuwenden. $G$ soll alle möglichen Züge eines Springers auf einem Schachfeld
modellieren. Jedem Springerzug - somit jeder Kante $e \in A$ - werden die Kosten $c(e) = 6$ zugewiesen. Der 
Startknoten $s$ des Springers liegt fix bei (C,4), der Zielknoten $t$ bei (A,1). Die Algorithmen A* und IDA* 
sollen jeweils mit drei parametrisierten, heuristischen Abstandsfunktionen $h_i: V \rightarrow \mathbb{R}$ 
durchgeführt werden:
\[ h_0(v) \quad = \quad 0 \]
\[ h_1(v) \quad = \quad \alpha_1\cdot\Big( x(t)-x(v) \quad+\quad y(t)-y(v) \Big) \]
\[ h_2(v) \quad = \quad \alpha_2\cdot max\{x(t)-x(v), y(t)-y(v)\} \]
Wobei $\alpha \geq 0$ ist. $x: V \rightarrow \mathbb{N}_0$ beschreibt den alphabetischen Koordinatenwert und 
$y: V \rightarrow \mathbb{N}_0$ den numerischen Koordinatenwert einer Position auf dem Schachfeld:
\[ x((C,4)) = 3, \quad y((C,4)) = 4 \]

\paragraph*{}
Insgesamt sind also sieben Algorithmendurchführungen (dreimal A*, dreimal IDA*, einmal BFS) auf einem Graphen 
mit 64 Knoten anzufertigen und schrittweise zu dokumentieren. Wegen der schieren Länge dieser Dokumentation 
entscheiden wir uns dafür, die Algorithmen und ihre Dokumentation in geeigneter Form zu implementieren. Als 
Sprache für dieses Vorhaben wählen wir C\#, weil ihre Runtime bereits über viele komfortable Datenstrukturen, 
wie z.B. generic sets, und mit LINQ (LanguageINtegratedQuery) über mächtige Transformations- und 
Filterfunktionen auf diesen Datenstrukturen verfügt. Insbesondere erlauben die generic sets eine Implementierung, 
die im Sinne der Lesbarkeit sehr nahe an den im Skript vorgestellten Pseudocode-Implementierungen liegt. Der 
Quellcode ist diesem Dokument nicht direkt beigefügt sondern wird separat als VisualStudio-Projektmappe übersandt.
Diese enthält neben aller Meta-Informationen, die zum Kompilieren des Projekts benötigt werden, auch eine für 
Windows 10 x64 kompilierte Executable-Datei, um das Programm direkt ausführen zu können. Das Programm erschließt 
sich zunächst den Graphen vom Startknoten (C,4) aus, führt dann alle sieben Algorithmen nacheinander aus und 
schreibt entsprechende Protokolle als CSV-Dateien in einen Ordner auf dem Desktop.
