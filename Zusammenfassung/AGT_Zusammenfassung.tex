\documentclass[10pt,a4paper,landscape]{article}
\usepackage[headheight=14pt,left=0.55cm,right=0.55cm,top=1.10cm,bottom=0.55cm,landscape,
headsep=2mm]{geometry}

\usepackage{lastpage}
\usepackage{fancyhdr}
\usepackage{multicol}
\usepackage[utf8]{inputenc}
\usepackage{graphicx}
\usepackage{wrapfig}
\usepackage{enumitem}
\usepackage{titlesec}
\usepackage{tabularx}
\usepackage[T1]{fontenc}
\usepackage{lmodern}
\usepackage{todonotes}
\usepackage{amsmath}
\usepackage{amssymb}
%\usepackage{showframe}


% Header
\pagestyle{fancy}
\fancyhead{}
\fancyfoot{}
\fancyhead[L]{Zusammenfassung Algorithmische Graphentheorie WiSe 2019/2020}
\fancyhead[R]{Seite $\thepage$ von $\pageref{LastPage}$}
\fancyheadoffset{0cm}

% Document
\setlength{\columnseprule}{0.5pt}
\setlength{\topskip}{10pt}
%\setlist{nosep}

\titleformat*{\section}{\normalsize\bfseries}
\titleformat*{\subsection}{\small\bfseries}
\titleformat*{\subsubsection}{\small\bfseries}
\titleformat*{\paragraph}{\bfseries}
\titleformat*{\subparagraph}{\bfseries}

\titlespacing*{\section}
{4pt}{4pt}{4pt}
\titlespacing*{\subsection}
{4pt}{4pt}{4pt}
\titlespacing*{\subsubsection}
{4pt}{4pt}{4pt}
\titlespacing*{\paragraph}
{4pt}{4pt}{8pt}

\newcolumntype{P}[1]{>{\centering\arraybackslash}p{#1}}
\newcolumntype{M}[1]{>{\centering\arraybackslash}m{#1}}

\makeatletter
\newcommand{\xRightarrow}[2][]{\ext@arrow 0359\Rightarrowfill@{#1}{#2}}
\makeatother

% Building blocks
\newcommand{\heading}[1]{\noindent\section*{\framebox[\columnwidth][l]{#1}}}
\newcommand{\subheading}[1]{\noindent\subsection*{\framebox[\columnwidth][l]{#1}}}
\newcommand{\subsubheading}[1]{\noindent\framebox[\columnwidth][l]{#1}}
\newcommand{\ccontent}[1]{\parbox{\columnwidth}{\centering{#1}}}

\newenvironment{allesInCode}{\ttfamily}{\par}


% Content
\begin{document}
	\begin{multicols*}{2}
        \normalsize

        
        \section{ Grundlagen }
            \subsection{ Graphen und Digraphen }
            
            \paragraph*{ Graph G }
            \[ G = (V,E) \]
            wobei $V$ eine endliche, nichtleere Menge an Knoten oder Ecken (Vertices) und $E$ eine Menge Kanten (Edges) 
            , gegeben durch ungeordnete Paare von Knoten $(v,w) v, w \in V$, ist.
            
            \paragraph*{ Digraph D }
            \[ D = (V,A)\]
            wobei $V$ eine endliche, nichtleere Menge an Knoten oder Ecken (Vertices) und $A$ eine Menge von gerichteten 
            Bögen (Arcs) $A \subseteq V \times V$ ist.

            \paragraph*{ Grad } eines Knotens $v$: \\
            $deg(v) :=$ Grad von v - Anzahl zu Knoten $v$ inzidenter (benachbarter) Kanten in einem Graphen \\
            $deg_{+}(v), deg_{in}(v) :=$ Eingangsgrad von $v$ - Anzahl aller auf $v$ zeigenden Bögen in einem Digraphen \\
            $deg_{-}(v), deg_{out}(v) :=$ Ausgangsgrad von $v$ - Anzahl aller von $v$ ausgehenden Bögen in einem Digraphen 

            \paragraph*{ Kantenzug/Kette/Pfad P }
            \[ P = (v_0, e_1, v_1, e_2, v_2, \dots, e_k, v_k), v_i \in V, e_i \in E \]
            ist eine Sequenz zusammenhängender Kanten. 
            \paragraph*{ Weg } ist ein Kantenzug, bei dem alle Kanten (paarweise) verschieden sind, also keine wiederholt wird.
            \paragraph*{ Kreis/Zyklus } ist ein geschlossener Weg mit identischem Start und Zielknoten.

            \paragraph*{ vollständiger Graph $K_{n}$ } ist der Graph aus $n$ Knoten, der jeden Knoten mit jedem anderen Knoten verbindet.
            \paragraph*{ bipartiter Graph G } $V$ kann in zwei disjunkte Teilmengen $V_1$ und $V_2$ aufgeteilt werden, sodass jede Kante einen 
            Endpunkt in $V_1$ und $V_2$ hat.
            \paragraph*{ Satz von König }
            Ein Graph ist genau dann bipartit, wenn er keinen Kreis ungerader Länge enthält.

            \subsection{ Bäume und Wälder }
            \paragraph*{ Wald $F$ } ist ein Graph, der keinen Kreis enthält.
            \paragraph*{ Baum $T$ in einem Graphen $G$ } ist ein zusammenhängender Wald. $T$ heißt aufspannend, wenn er alle Knoten von $G$ enthält. 

            \subsection{ Handshaking-Lemma }
            In einem Graphen $G = (V, E)$ gilt:
            \[ \sum_{v\in V} deg(v) = 2|E| \]
            Daraus folgt implizit, dass die Anzahl der Knoten mit ungeradem Grad gerade ist.

            \subsection{ Planare Graphen }
            Ein planarer Graph kann auf einer zweidimensionalen Fläche so kreuzungsfrei gezeichnet werden. Jeder Kreis, 
            jeder Baum und der vollständige Graph $K_4$ sind planar. Der vollständige Graph $K_5$ und der vollständige 
            bipartite Graph $K_{3,3}$ sind nicht planar.
            \paragraph*{ Satz von Kurtowski } ein Graph ist genau dann nicht planar, wenn er durch Kontraktion von Kanten 
            in den $K_5$ oder den $K_{3,3}$ überführt werden kann.
            \paragraph*{ Eulersche Polyederformel }
            \[ n - m + f = 2 \]
            wobei $n$ die Anzahl der Knoten, $m$ die Anzahl der Kanten und $f$ die Anzahl der Flächen in einem 
            zusammenhängenden planaren Graphen ist. \\
            Für nicht zusammenhängende Graphen gilt
            \[ n - m + f = k + 1 \]
            wobei $k$ die Anzahl der Zusammenhangskomponenten ist. \\
            Aus der Polyederformel folgt unmittelbar für planare Graphen aus 3 oder mehr Knoten:
            \[ m \leq 3n - 6 \]
            \[ f \leq 2n - 4 \]
            Ist der Graph außerdem noch bipartit gilt für die Anzahl der Kanten:
            \[ m \leq 2n - 4 \]
            Insbesondere hat jeder planare Graph mindestens einen Knoten von Grad höchstens 5.

            \subsection{ Topologische Sortierung }
            Eine topologische Sortierung eines Digraphen $D = (V,A)$ ist eine injektive Abbildung 
            $f:V\Rightarrow \mathbb{N}$, sodass gilt:
            \[ (v,w)\in A \Rightarrow f(v) < f(w) \]
            Ein Digraph hat genau dann eine topologische Sortierung, wenn er keinen gerichteten Kreis enthält.
            \newpage
            \paragraph*{ TopSort }
            \begin{verbatim}
    input: ein beliebiger Digraph
    ouput: eine topologische Sortierung f oder die Aussage, 
           dass ein Kreis vorliegt

    index = 0

    while( es ist ein Knoten v mit Eingangsgrad > 0 vorhanden )
        entferne v und aktualisiere die Eingangsgrade seiner Nachbarn
        setze die topologische Sortierung f von v auf index
        erhöhe index um 1

    if( es sind noch Knoten übrig )
        return false, es liegt ein Kreis vor
    else 
        return topologische Sortierung f
            \end{verbatim}

        \section{ Suche in Graphen }
            \subsection{ Breitensuche (BFS) }

            \subsection{ Tiefensuche (DFS) }


        \section{ Minimal aufspannende Bäume }
            \subsection{ Definition } \todo{ umbenennen }

            \subsection{ Kruskal }

            \subsection{ Prim }


        \section{ Kürzeste Wege }
            \subsection{ Bellmansches Optimierungskriterium }

            \subsection{ Dijkstra }

            \subsection{ Moore-Bollman-Ford }

            \subsection{ A* }

            \subsection{ Dynamische Programmierung - Rucksackproblem }

        \section{ Flüsse in Netzwerken }
            \subsection{ Max-Flow-Min-Cut-Theorem }\subsection{  }

            \subsection{ Edmonds-Karp }

            \subsection{ Push-Relabel }


        \section{ Matchings }
            \subsection{ Matchings via augmentierende Wege }

            \subsection{ Blossom-Shrinking }


        \section{ Euler- \& Hamiltonkreise }
            \subsection{ Eulertour \& -kreis}

            \subsection{ Chinesisches Postboten-Problem (CPP) }

            \subsection{ Hamiltonkreis }

            \subsection{ Travelling-Salesman-Problem (TSP) }


        \section{ Färbung von Graphen }
            \subsection{ Färbung planarer Graphen }

            \subsection{ Heuristiken zur Graphenfärbung }


    \end{multicols*}
\end{document}